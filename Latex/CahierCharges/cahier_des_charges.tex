\documentclass[a4paper,12pt]{article}

\usepackage[utf8]{inputenc}
\usepackage[a4paper, left=3cm, right=3cm, top=3cm, bottom=3cm]{geometry}
\usepackage[frenchb]{babel}
\usepackage{default}
\usepackage{pslatex}
\usepackage{graphicx}
\usepackage{algorithmic}
\usepackage{multicol}
\usepackage{amsmath}
\usepackage{amssymb}
\usepackage{textcomp}
\usepackage{pgf}
\usepackage{tikz}
\usepackage{pgfplots}
\usepackage{capt-of}
\usepackage{esvect}
\usepackage[T1]{fontenc}

\title{Présentation et cahier des charges du projet Pain Chocolatine}
\author{Wagner Robin, Clerc Gregory, Pleinet Estelle, Porteries Tristan}

\begin{document}

\maketitle

\tableofcontents

\section{Présentation}

\subsection{Exemple}

\section{Actions de l'application}

L'application possède deux vues, une pour enregistrer et consulter des réservations de la part du client, et une seconde pour créer des produits de la part du commerçant et lister les réservations actuelles. Ces deux vues comportent une liste d'actions trié par ordre de priorité. Elles se présentent sous la forme suivante:

\subsection{Vue client}

\subsubsection{Enregistrer un compte et se connecter}

Un client doit pouvoir créer un compte nécessitant les informations suivantes~;
\begin{itemize}
	\item email~: pour pouvoir contacter le client en cas de problèmes~;
	\item nom~: nom pour lequel le client va pouvoir se connecter et retirer son produit auprès d'un commerçant~;
	\item mot de passe~: mot de passe pour se connecter à l'application.
\end{itemize} \

Il suffit du nom et du mot de passe pour se connecter à l'application.

\subsubsection{Lister les commerces}

Cette action présente une liste de tous les commerces enregistrés sur l'application, le client peut cliquer sur un des éléments pour voir la description du commerce ou procéder à des opérations concernant ce commerce.

\subsubsection{Reserver des produits}

Dans le contexte d'un commerce sélectionné, le client peux voir une liste de produits disponibles selon les contraintes et en réserver un ou plusieurs avec deux champs supplémentaires à remplir~:

\begin{itemize}
	\item horaire : l'horaire de retrait du produit parmi ceux proposés par le commerçant, par exemple 12h et 18h~;
	\item quantité : la quantité du produit reservé dans le cadre des limitations du commerçant pour son commerce et par client.
\end{itemize} \

Les produits nécessitant un temps minimum avant réservation ne peuvent pas être reservé tout comme pour les produits ayant atteint leur limitation (limitation du commerce ou par client) seront visibles mais impossibles à réserver.

\subsubsection{Liste les réservations}

Après plusieurs réservations le client à la possibilité d'afficher ses réservations, où chaque réservation porte les informations suivantes~:

\begin{itemize}
	\item lien et nom du commerce~;
	\item lien et nom du produit~;
	\item quantité~;
	\item horaire~;
	\item état du produit : si le produit a déjà été récupéré, celui ci indique «achevé».
\end{itemize}


\subsubsection{Supprimer une reservation}

Depuis la page listant les réservations, le client peut supprimer une réservation si le commerce le permet.

\subsection{Vue commerçant}

\subsubsection{Enregistrer un commerce}

Un commerçant doit pouvoir enregistrer son commerce avec les informations suivantes~:

\begin{itemize}
	\item adresse~: l'adresse du commerce~;
	\item nom~: le nom du commerce~;
	\item type / description~: une description textuelle du commerce~;
	\item contact~: un numéro de téléphone du commerce ainsi qu'une adresse email~;
	\item mot de passe~: mot de passe pour accéder à la gestion des réservations.
\end{itemize}

Tout comme les clients, le commerçant se connecte en utilisant le nom de son commerce et son mot de passe.

\subsubsection{Lister les produits}

Un commerçant peut lister tous les produits déjà disponibles dans son commerce.

\subsubsection{Lister les réservations quotidiennes}

Un commerçant doit pouvoir lister les réservations actuelles du jour même avec les informations suivantes~:

\begin{itemize}
	\item nom du client~;
	\item lien et nom du produit~;
	\item quantité~;
	\item horaire~;
	\item état du produit~: si le produit a déjà été récupéré, celui ci indique «achevé».
\end{itemize}

\subsubsection{Ajouter des produits}

Le commerçant peut enregistrer un produit avec les données suivantes~:

\begin{itemize}
	\item nom du produit~;
	\item description du produit~;
	\item prix~;
	\item horaires de retrait~;
	\item quantité maximale pour le commerce à chaque horaire~: au delà d'un nombre de réservations cumulé égal, aucun client ne peux réserver ce produit~;
	\item quantité maximale par client à chaque horaire~: au delà d'un nombre de réservation égal, le client ne peut plus réserver ce produit~;
	\item temps de réservation 
\end{itemize}


\subsubsection{Supprimer des produits}

Depuis la vue de tous les produits, un commerçant peut supprimer un de ses produits, il sera impossible à réserver mais les réservations actuelles seront conservées jusqu'à leur achèvement.

\subsubsection{Marque des reservations comme achevées}

Un commerçant peut depuis la liste des reservations actuelles marquer une réservation comme achevée lorsque celle ci a été donnée au client.

\section{Modèle de données}

Aprés une première analyse des actions, quatre types on été produits~:
\begin{itemize}
	\item Client~: un compte client~;
	\item Commerce~: un compte d'un commerce~;
	\item Produit~: les données d'un produit~;
	\item Reservation~: les données de réservation~;
\end{itemize}

Tous ces types sont représentés dans des bases de données dans au moins une table différente.

\subsubsection{Client}

Le type Client contient les champs suivants~:
\begin{itemize}
	\item id (integer)~;
	\item nom (string)~;
	\item email (string)~;
	\item mot de passe.
\end{itemize} \

Les clients sont tous enregistrés dans la même table.

\subsubsection{Commerce}

Le type Commerce contient les champs suivants~:
\begin{itemize}
	\item id (integer)~;
	\item nom (string)~;
	\item description (string)~;
	\item contact téléphone (integer)~;
	\item contact email (string)~;
	\item mot de passe~;
	\item nom table produit (string).
\end{itemize} \

Les commerces sont tous enregistrés dans la même table.

\subsubsection{Produit}

Le type Produit contient les champs suivants~:
\begin{itemize}
	\item id (integer)~;
	\item id commerce (integer)~;
	\item nom (string)~;
	\item description (string)~;
	\item prix (float)~;
	\item horaires (array date)~;
	\item quantité maximale cumulé (integer)~;
	\item quantité maximale par client (integer).
\end{itemize} \

Les produits d'un même commerce sont enregistrés dans des tables différentes.

\subsubsection{Réservation}

Le type Réservation contient les champs suivants~:
\begin{itemize}
	\item id (integer)~;
	\item id client (integer)~;
	\item id produit (integer)~;
	\item horaire (date)~;
	\item quantité (integer)~;
	\item état.
\end{itemize} \

Les réservation sont toutes enregistrées dans la même table.

\section{Modèle d'interaction}

Différentes fonctions peuvent être décrites entre ces types, ces fonctions permettent en général la consultation, l'ajout et la suppression de produits et réservations comme exposés dans la description des actions précédentes.

Leur implémentation interagit directement avec la base de données depuis le PHP.

\subsubsection{De Commerce vers Produit}

\begin{itemize}
	\item $obtenirProduits(Commerce) \rightarrow [Produit]$~: donne la liste des produits pour un commerce~;
	\item $ajouterProduit(Commerce, Produit)$~: ajoute un produit préalablement construit~;
	\item $supprimerProduit(Commerce, idproduit)$~: supprimer un produit par son identifiant.
\end{itemize}

\subsubsection{De Client vers Réservation}

\begin{itemize}
	\item $obtenirReservations(Client) \rightarrow [Reservation]$~: donne la liste des réservations pour un client~;
	\item $ajouterReservation(Client, Produit, horaire, quantite) \rightarrow bool$~: tente d'ajouter une réservation, si la quantité est supérieur aux contraintes, renvoi faux~;
	\item $supprimerReservation(Client) \rightarrow bool$~: tente de supprimer une réservation, renvoi faux en cas de non permission.
\end{itemize}

\subsubsection{De Commerce vers Réservation}

\begin{itemize}
	\item $obtenirReservations(Commerce) \rightarrow [Reservation]$~: donne la liste des réservations actuelles pour un commerce~;
	\item $validerReservation(Reservation)$~: change l'état d'une réservation en achevée.
\end{itemize}


\end{document}
