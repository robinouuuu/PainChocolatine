\documentclass[a4paper,12pt]{article}

\usepackage[utf8]{inputenc}
\usepackage[a4paper, left=3cm, right=3cm, top=3cm, bottom=3cm]{geometry}
\usepackage[frenchb]{babel}
\usepackage{default}
\usepackage{pslatex}
\usepackage{graphicx}
\usepackage{algorithmic}
\usepackage{multicol}
\usepackage{amsmath}
\usepackage{amssymb}
\usepackage{textcomp}
\usepackage{pgf}
\usepackage{tikz}
\usepackage{pgfplots}
\usepackage{capt-of}
\usepackage{esvect}
\usepackage[T1]{fontenc}

\title{Présentation et cahier des charges du projet Pain Chocolatine}
\author{Wagner Robin, Clerc Gregory, Pleinet Estelle, Porteries Tristan}

\begin{document}

\maketitle

\tableofcontents

\section{Présentation}

\subsection{Exemple}

\section{Actions de l'application}

L'application possède deux vues, une pour enregistrer et consulter des réservation de la part du client, et une seconde pour créer des produits de la part du commerçant et lister les reservations actuelles. Ces deux vues comportent une liste d'actions triés par ordre de priorité et décritent par la suite.

\subsection{Vue client}

\subsubsection{Enregistrer un compte et se connecter}

Un client doit pouvoir créer un compte nécessitant les informations suivante~;
\begin{itemize}
	\item email~: pour pouvoir contacter le client en cas de problèmes~;
	\item nom~: nom pour lequel le client va pouvoir se connecter et retirer son produit au près d'un commerçant~;
	\item mot de passe~: mot de passe pour se connecter à l'application.
\end{itemize} \

Il suffit du nom et du mot de passe pour se connecter à l'application.

\subsubsection{Lister les commerces}

Cette action presente une liste de tous les commerces enregistrés sur l'application, le client peut cliquer sur un des éléments pour voir la description du commerce ou procédé à des opérations concernant ce commerce.

\subsubsection{Reserver des produits}

Dans le contexte d'un commerce séléctionné, le client peux voir une liste de produits disponible selon les contraintes et en reservé un ou plusieurs avec deux champs supplémentaires à remplir~:

\begin{itemize}
	\item horaire : l'horaire de retrait du produit parmis ceux proposé par le commerçant, par exemple 12h et 18h~;
	\item quantité : la quantité du produit reservé dans le cadre des limitations du commerçant pour son commerce et par client.
\end{itemize} \

Les produits nécessitant un temps minimum avant reservation ne peuvent pas être reservé tout comme pour les produits ayant atteind leur limitation (limitation du commerce ou par client) seront visible mais impossible à reserver.

\subsubsection{Liste les réservations}

Après plusieurs réservations le client à la possibilité d'afficher ses réservations, où chaque reservation porte les informations suivantes~:

\begin{itemize}
	\item lien et nom du commerce~;
	\item lien et nom du produit~;
	\item quantité~;
	\item horaire~;
	\item état du produit : si le produit a déjà été récupéré celui ci indique «achévé».
\end{itemize}


\subsubsection{Supprimer une reservation}

Depuis la page listant les reservations, le client peut supprimer une reservation si le commerce le permet.

\subsection{Vue commerçant}

\subsubsection{Enregistrer un commerce}

Un commerçant doit pouvoir enregistrer son commerce avec les informations suivantes~:

\begin{itemize}
	\item adresse~: l'adresse du commerce~;
	\item nom~: le nom du commerce~;
	\item type / description~: une description textuelle du commerce~;
	\item contact~: un numéro de téléphone du commerce ainsi qu'une adresse email~;
	\item mot de passe~: mot de passe pour acceder à la gestion des réservations.
\end{itemize}

Tout comme les clients, le commerçant se connecte en utilisant le nom de son commerce et son mot de passe.

\subsubsection{Lister les produits}

Un commerçant peut lister tout les produits déjà disponible dans son commerce.

\subsubsection{Lister les reservations quotidienne}

Un commerçant doit pouvoir lister les réservations actuelles du jour même avec les information suivantes~:

\begin{itemize}
	\item nom du client~;
	\item lien et nom du produit~;
	\item quantité~;
	\item horaire~;
	\item état du produit~: si le produit a déjà été récupéré celui ci indique «achévé».
\end{itemize}

\subsubsection{Ajouter des produits}

Le commerçant peut enregistrer un produits avec les données suivantes~:

\begin{itemize}
	\item nom du produit~;
	\item description du produit~;
	\item prix~;
	\item horaires de retrait~;
	\item quantité maximal pour le commerce à chaque horaire~: au dela d'un nombre de reservations cumulé égale, aucun client ne peux reserver ce produit~;
	\item quantité maximal par client à chaque horaire~: au dela d'un nombre de reservation égale le client ne peut plus reserver ce produit~;
	\item temps de reservation 
\end{itemize}


\subsubsection{Supprimer des produits}

Depuis la vue de tous le produits, un commerçant peux supprimer un des ses produits, il sera impossible à reserver mais les reservations actuelles seront conservées jusqu'à leur achèvement.

\subsubsection{Marqué des reservations comme achevées}

Un commerçant peut depuis la liste des reservations actuelles marqué une reservation comme achevée lorsque celle ci à été donnée au client.

\section{Modèle de donnée}

Aprés une première analyse des actions, quatre types on été produits~:
\begin{itemize}
	\item Client~: un compte client~;
	\item Commerce~: un compte d'un commerce~;
	\item Produit~: les données d'un produit~;
	\item Reservation~: les données de reservation~;
\end{itemize}

Tous ces types sont représentés dans des base de données dans au moins une table différente.

\subsubsection{Client}

Le type Client contient les champs suivants~:
\begin{itemize}
	\item id (integer)~;
	\item nom (string)~;
	\item email (string)~;
	\item mod de passe.
\end{itemize} \

Les clients sont tous enregistrés dans la même table.

\subsubsection{Commerce}

Le type Commerce contient les champs suivants~:
\begin{itemize}
	\item id (integer)~;
	\item nom (string)~;
	\item description (string)~;
	\item contact téléphone (integer)~;
	\item contact email (string)~;
	\item mot de passe~;
	\item nom table produit (string).
\end{itemize} \

Les commerces sont tous enregistrés dans la même table.

\subsubsection{Produit}

Le type Produit contient les champs suivants~:
\begin{itemize}
	\item id (integer)~;
	\item id commerce (integer)~;
	\item nom (string)~;
	\item description (string)~;
	\item prix (float)~;
	\item horaires (array date)~;
	\item quantité maximale cumulé (integer)~;
	\item quantité maximale par client (integer).
\end{itemize} \

Les produits d'un même commerce sont enregistrés dans des tables différentes.

\subsubsection{Reservation}

Le type Reservation contient les champs suivants~:
\begin{itemize}
	\item id (integer)~;
	\item id client (integer)~;
	\item id produit (integer)~;
	\item horaire (date)~;
	\item quantité (integer)~;
	\item état.
\end{itemize} \

Les reservation sont toutes enregistrées dans la même table.

\section{Modèle d'interaction}

Différentes fonctions peuvent être décrites entre ces types, ces fonctions permettent en générale la consulation, l'ajout et la suppression de produits et reservations comme exposé dans la description des actions précedement.

Leur implémentation intérragit directement avec la base de données depuis le PHP.

\subsubsection{De Commerce vers Produit}

\begin{itemize}
	\item $obtenirProduits(Commerce) \rightarrow [Produit]$~: donne la liste des produits pour un commerce~;
	\item $ajouterProduit(Commerce, Produit)$~: ajoute un produit préalablement construit~;
	\item $supprimerProduit(Commerce, idproduit)$~: supprimer un produit par son identifiant.
\end{itemize}

\subsubsection{De Client vers Reservation}

\begin{itemize}
	\item $obtenirReservations(Client) \rightarrow [Reservation]$~: donne la liste des reservations pour un client~;
	\item $ajouterReservation(Client, Produit, horaire, quantite) \rightarrow bool$~: tente d'ajouter une réservation, si la quantité est supérieur aux contraintes, renvoi faux~;
	\item $supprimerReservation(Client) \rightarrow bool$~: tente de supprimer une reservation, renvoi faux en cas de non permission.
\end{itemize}

\subsubsection{De Commerce vers Reservation}

\begin{itemize}
	\item $obtenirReservations(Commerce) \rightarrow [Reservation]$~: donne la liste des reservations actuelles pour un commerce~;
	\item $validerReservation(Reservation)$~: change l'état d'une reservation en achevée.
\end{itemize}


\end{document}
